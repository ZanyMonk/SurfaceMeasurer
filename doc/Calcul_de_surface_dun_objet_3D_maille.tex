\documentclass[12pt]{article}
\usepackage{natbib}
\usepackage{url}
\usepackage[utf8]{inputenc}
\usepackage{amsmath}
\usepackage{graphicx}
\usepackage{parskip}
\usepackage{fancyhdr}

\title{Calcul de surface d'un objet 3D maillé}
\author{Lucien Aubert $\newline$ Thibaut Jallois}

\makeatletter
\let\thetitle\@title
\makeatother

\pagestyle{fancy}
\fancyhf{}
\chead{\thetitle}
\cfoot{\thepage}

\begin{document}

%%%%%%%%%%%%%%%%%%%%%%%%%%%%%%%%%%%%%%%%%%%%%%%%%%%%%%%%%%%%%%%%%%%%%%%%%%%%%%%%%%%%%%%%%

\begin{titlepage}
	\centering
    \vspace*{0.5 cm}
    \includegraphics[scale = 0.15]{logo-amu.png}\\[1.0 cm]	% University Logo
    \textsc{\LARGE Aix-Marseille Université}\\[2.0 cm]
	\textsc{\Large M3101- Systèmes d'exploitation}\\[0.5 cm]
	\rule{\linewidth}{0.2 mm} \\[0.4 cm]
	{ \huge \bfseries \thetitle}\\
	\rule{\linewidth}{0.2 mm} \\[1.5 cm]

	\begin{minipage}{0.4\textwidth}
		\begin{flushleft} \large
			\emph{Auteurs :}\\
			\theauthor
			\end{flushleft}
			\end{minipage}~
			\begin{minipage}{0.4\textwidth}
			\begin{flushright} \large
			\emph{Professeur :} \\
			Romain Raffin
		\end{flushright}
	\end{minipage}\\[2 cm]

	{\large \thedate}\\[2 cm]

	\vfill

\end{titlepage}

\tableofcontents
\pagebreak

\section{Introduction}
L'objectif consiste en l'optimisation, par parallèlisation, du calcul de la surface d'un objet 3D maillé (triangles) au format OFF\cite{OFF} à l'aide de la formule de Héron\cite{heron}.

Le programme implémente trois algorithmes
\begin{itemize}
	\item Classique, séquentiel
	\item Avec \texttt{pthread}\cite{pthreads}, parallélisé
	\item Avec \texttt{OpenMP}, parallélisé également
\end{itemize}

\section{Environnement d'expérimentation}
La phase de test s'est déroulée sur trois machines dont voici les configurations

\subsection{Machine 1}
\texttt{Intel i7-3612QM 2.10GHz, 8 CPU, 4 cœurs, L1 64K, L2 256K, L3 6144K}

\texttt{12Go RAM DDR3 800MHz}

\subsection{Machine 2}
Machine Thibaut

\subsection{Machine 3}
Machine IUT

\section{Algorithmique et implémentation}
	\subsection{Algorithme séquentiel}
		\subsubsection{Conception}
		\subsubsection{Résultats}

	\subsection{Algorithmes parallèles}
		\subsubsection{Threads}
			\paragraph{Conception}
			\paragraph{Résultats}
			
		\subsubsection{OpenMP}
			\paragraph{Conception}
			\paragraph{Résultats}

\section{Conclusion}
% Conclusion et conjecture

\newpage
\bibliographystyle{unsrt}
\bibliography{biblio}

\end{document}
